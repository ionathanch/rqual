\documentclass{article}
\usepackage[utf8]{inputenc}
\usepackage[T1]{fontenc}
\usepackage[left=1.5in, right=1.5in, top=1.3in, bottom=1.3in]{geometry}
\usepackage{mlmodern}
\usepackage{amsmath,amssymb}
\usepackage[english]{babel}
\usepackage[square, authoryear]{natbib}
\usepackage[colorlinks,citecolor=blue]{hyperref}
\usepackage[capitalize,noabbrev,nameinlink]{cleveref}
\usepackage[margin=2\parindent]{caption}
\usepackage{mathpartir,mathtools,stmaryrd}
\usepackage{enumitem,doi,xspace,minted,tikz-cd}

\title{\textbf{Internalizing Extensions \\ in Lattices of Type Theories}}
\author{Jonathan Chan}
\date{21 November 2024}

\bibliographystyle{ACM-Reference-Format}

\setlength\parskip{1ex plus 0.1em minus 0.2em}

% If needed, define \crefrangeformat, \crefmultiformat, \crefrangemultiformat
\crefformat{question}{#2Question~#1#3}
\Crefformat{question}{#2Question~#1#3}

\newcommand{\ie}{\textit{i.e.}\@\xspace}
\newcommand{\eg}{\textit{e.g.}\@\xspace}
\newcommand{\apost}{\textit{a posteriori}\@\xspace}
\newcommand{\kw}[1]{\mathsf{#1}}
\newcommand{\code}[1]{\texttt{#1}}
\newcommand{\HH}{\mathsf{H}}
\newcommand{\LL}{\mathsf{L}}
\renewcommand{\SS}{\mathsf{S}}

\begin{document}
\maketitle

\begin{abstract}
  While many proof assistants are founded upon common theoretical ground,
  they will also feature further extensions and axioms
  that augment their reasoning power.
  The individual theories based on each extension are individually sound,
  but how type checkers handle their interactions at compile time
  is not formally described.
  Furthermore, the way proof assistants currently enable extensions
  can be too coarse and imprecise when it comes to
  guarantees about the features a top-level definition may use.

  A first step towards a framework of extensions could use
  the Dependent Calculus of Indistinguishability (DCOI)~\citep{dcoi}
  to formalize granular and precise extension tracking.
  DCOI is a type system with dependency tracking,
  where terms and variables are assigned dependency levels alongside their types.
  These dependency levels form a lattice that describes
  which levels are permitted to access what.
  This report explores how extensions could correspond to dependency levels,
  and how the lattice would describe how extensions are permitted to interact.
\end{abstract}

\section{Introduction}

Dependent type theories form the theoretical foundations of many proof assistants
used for automating and mechanizing theorems in areas ranging from pure mathematics
to modelling real programming languages.
Flavours of theories such as Martin-L\"of Type Theory (MLTT)~\citep{mltt}
and the Calculus of Inductive Constructions (CIC)~\citep{cic}
form the basis of some of the most widely used proof assistants.
By the Curry--Howard correspondence,
where propositions correspond to types and proofs correspond to terms,
the core of proof assistants are language implementations.
As such, they share many of the same components
as implementations of typical programming languages,
such as a parser, a type checker, and a compiler.

% can assume that ppl know about proof assistant extensions
Like many language implementations,
proof assistants will incorporate optional language features
that influence the behaviour of its inner components.
For programming languages, these features may be compiler optimizations,
static analyses that restrict the language to enable these optimizations,
or type system extensions that increase what the type checker accepts.
Because the primary use case of proof assistants is to type check mechanized proofs,
their language extensions are largely the latter,
each of which alters the underlying type theory.

Consequently, the effects of extensions in proof assistants are nonlocal,
as opposed to, say, marking a function as tail recursive,
which does not affect the behaviour of functions around it.
Using a particular type theoretic extension in a module, for instance,
changes the reasoning principles that are or aren't allowed in that module,
which can infect other modules that import it.
Furthermore, there exist extensions that are mutually incompatible
because together they would violate logical consistency of the type theory.
In \cref{sec:extensions},
I describe a few such extensions found in select proof assistants,
and the ways some combinations are incompatible.

Thus when proof assistants provide mechanisms for language extensions,
they are careful to track the usage of these extensions to rule out inconsistencies.
The tracking done by the type checker is \emph{external} to the type system:
within the language itself,
one cannot assert that a definition is permitted to depend on a particular extension,
nor that it is guaranteed \emph{not} to use it.
Furthermore, while each extension may have been independently verified to be consistent,
there are no formal descriptions of interactions between parts of code
that use different sets of extensions.
Ultimately, what we are missing is a framework for describing
fine-grained control of proofs and programs across multiple type theories.

Luckily, there do exist frameworks for describing fine-grained control
of proofs and programs across multiple \emph{dependency levels},
which exist within the same type system,
but are stratified by some hierarchical structure
of what can be thought of intuitively as permission levels.
In particular, I have worked on the Dependent Calculus of Indistinguishability
(DCOI)~\citep{dcoi} and its variant DCOI$^\omega$~\citep{dcoi-omega},
the latter of which is a type theory with dependency tracking,
whose key properties I describe in \cref{sec:dcoi}.

I propose that DCOI can be used as a basis for a framework
which internalizes extension tracking by stratifying
their corresponding type theories into hierarchies of dependency levels,
where compatibility between extensions maps to the intuitive notion of permission.
In \cref{sec:lattice}, I speculate on the details of this mapping,
lay out the objectives for such a framework,
and list possible first steps towards accomplishing them.
There is much prior and related work that this project relies on and relates to,
which I divide into work I have personally contributed to (\cref{sec:prior})
and other work in this space (\cref{sec:related}).

\section{Proof Assistant Extensions in Practice} \label{sec:extensions}

To look at extensions in practice,
let us focus on three popular proof assistants:
Rocq~\citep{coq}, Agda~\citep{agda}, and Lean~\citep{lean}.
Broadly speaking, they are all based on variants of MLTT or CIC,
and have dependent functions, type universes, and inductive types,
which are sufficient to encode a wide variety of logical propositions and proofs.

Each of these proof assistants, however,
include features that extend the power of their foundations;
below are a few notable extensions.

\begin{itemize}[noitemsep,topsep=0pt]
  \item \textbf{Impredicativity}: Rocq and Lean, being based on CIC,
    feature a universe $\kw{Prop}$ of propositions such that
    a quantification (\ie dependent function type) $\forall (x : A) \mathpunct{.} B$
    is a proposition if $B$ is a proposition,
    regardless of the universe in which $A$ lives,
    which may be larger than $\kw{Prop}$.
    Inductive types may also be defined in $\kw{Prop}$,
    which permits its constructors to have argument types
    in universes larger than $\kw{Prop}$.
    Such inductives are said to be \emph{large}.
  \item \textbf{Definitional proof irrelevance}:
    The inhabitants of propositions in Lean are definitionally equal
    and thus treated as interchangeable during type checking;
    these propositions are said to be \emph{strict}.
    Rocq has a separate $\kw{SProp}$ universe of proof-irrelevant propositions,
    and Agda has a predicative hierarchy $\kw{Prop}_i$ of such universes \citep{sprop}.
  \item \textbf{Uniqueness of identity proofs} (UIP):
    Agda's default pattern matching behaviour admits
    that inhabitants of the same propositional equality
    are themselves propositionally equal,
    which is proven via defining \textbf{Axiom K}~\citep{axiomk},
    a computational eliminator for propositional equalities of the form $a \equiv a$.
  \item \textbf{Strong elimination}:
    Also known as \emph{large} elimination,
    an element of an inductive datatype can be destructed or eliminated
    into a type at a larger universe.
    That is, a term whose type is in $\kw{Type}_i$
    can be eliminated to return a term whose type is in $\kw{Type}_j$
    for some $j > i$.
    For proofs of propositions in $\kw{Prop}$,
    large elimination corresponds to eliminating into any non-proposition.
\end{itemize}

Each of these increase expressivity by allowing more types to be stated and inhabited.
Impredicativity allows for self-referential propositions by quantifying over $\kw{Prop}$.
For instance, given a proof that all propositions imply their double negation,
$$\mathit{dn}: \forall (P : \kw{Prop})\mathpunct{.} P \to \neg \neg P,$$
the double negation of this proposition itself holds as well.
$$\mathit{dn} \; (\forall (P : \kw{Prop})\mathpunct{.} P \to \neg \neg P) \; \mathit{dn} : \neg \neg (\forall (P : \kw{Prop})\mathpunct{.} P \to \neg \neg P)$$
In the predicative setting, a quantification over a universe $\kw{Type}_0 : \kw{Type}_1$
itself has type $\kw{Type}_1$, and in particular,
$$\Pi (A : \kw{Type}_0)\mathpunct{.} A \to \neg \neg A : \kw{Type}_1,$$
so an element of this type may not be applied to the type itself.

Definitional proof irrelevance is useful to avoid having to prove equalities explicitly.
Consider a relation on two naturals asserting the usual less-than relation,
along with a type of bounded naturals from which the natural contained can be recovered.
%
\begin{align*}
  \cdot \le \cdot &: \kw{Nat} \to \kw{Nat} \to \kw{Prop} \\
  \mathit{BNat} &: \kw{Nat} \to \kw{Type}_0 \\
  \mathit{bNat} &: \Pi (n \; m : \kw{Nat})\mathpunct{.} n \le m \to \mathit{BNat} \; m \\
  \mathit{getNat} &: \Pi (m : \kw{Nat})\mathpunct{.} \mathit{BNat} \; m \to \kw{Nat} \\
  \mathit{getNat} &\; m \; (\mathit{bNat} \; n \; m \; p) \rightsquigarrow n
\end{align*}
%
A desirable property of bounded naturals is that
two bounded naturals are equal if their contained naturals are equal.
%
\begin{align*}
  \mathit{eqBNat} &: \forall (m : \kw{Nat}) (b_1 b_2 : \kw{BNat} \; m)\mathpunct{.}
    \mathit{getNat} \; m \; b_1 \equiv \mathit{getNat} \; m \; b_2 \to b_1 \equiv b_2
\end{align*}
%
Trying to prove this by destructing the bounded naturals as
$(\mathit{bnat} \; n_1 \; m \; p_1)$ and $(\mathit{bnat} \; n_2 \; m \; p_2)$,
while we have an equality $e : n_1 \equiv n_2$ by reduction of $\mathit{getNat}$,
we do \emph{not} have a proof of $p_1 \equiv p_2$%
\footnote{Technically, this requires a proof of the equality
where $p_1$ has been transported across the equality $e$
so that it has the same type as $p_2$.}.
Depending on how $\cdot \le \cdot$ is implemented,
it may be possible to prove propositionally that any two inequality proofs are equal.
Alternatively, if propositions are definitionally proof irrelevant,
the inequality proofs can effectively be ignored,
and rewriting the goal by $e$ is sufficient.

UIP is similarly useful to avoid reasoning about equalities between equalities
when the only canonical proof of an equality is reflexivity,
especially in settings without proof irrelevance.
While UIP does augment the reasoning power of the type theory,
there are inductive types whose equality proofs are already propositionally equal.
In particular, if a type has decidable equality,
\ie $(x \equiv y) \vee \neg (x \equiv y)$ for any given $x, y$ of that type,
then its equalities are themselves equal~\citep{hedberg}.

Naturally, not all types are proof-irrelevant,
such as the booleans, since $\kw{true}$ and $\kw{false}$ are distinct.
However, proving $\neg (\kw{true} \equiv \kw{false})$ requires strong elimination
to lift the booleans to propositions of truthhood $\top$ and falsehood $\bot$.
%
\begin{align*}
  \mathit{lift} &: \kw{Bool} \to \kw{Prop} \\
  \mathit{lift} &\coloneqq \lambda b \mathpunct{.} \kw{if} \; b \; \kw{then} \; \top \; \kw{else} \; \bot
\end{align*}
%
This is a strong elimination because it returns a type,
or equivalently because its return type is a universe.
To complete the proof,
given a proof of $\kw{true} \equiv \kw{false}$,
$\top \equiv \bot$ holds by congruence over $\mathit{lift}$,
across which the trivial proof $\kw{tt} : \top$ can be transported to $\bot$.

Some of these extensions are hidden behind an option flag.
To use strict $\kw{Prop}$,
Rocq requires the flag \code{Allow StrictProp},
while Agda requires the option \code{\{-\# OPTIONS -{}-prop \#-\}}.
While not inherently part of Rocq's type theory,
Axiom K is axiomatized in the standard library as \code{Coq.Logic.Eqdep.eq\_rect\_eq}.
Many other axioms are included in standard libraries
to further augment reasoning power;
below are a few notable examples.

\begin{itemize}[noitemsep,topsep=0pt]
  \item \textbf{Function extensionality}
    propositionally equates two functions if they are pointwise equal;
    in Lean as \code{funext} and in Rocq as \code{functional\_extensionality}.
  \item \textbf{Propositional extensionality}
    propositionally equates two propositions if they are bi\"implicated;
    in Lean as \code{propext} and in Rocq as \code{propositional\_extensionality}.
    Propositional proof irrelevance is a consequence of propositional extensionality.
  \item \textbf{Univalence}
    asserts an equivalence between propositional equality and equivalence,
    \ie given two types $A, B$,
    the equivalence $(A \equiv B) \simeq (A \simeq B)$ holds~\citep{hott}.
    There are several ways to define equivalence;
    the idea is that it captures a propositionally proof-irrelevant isomorphism.
  \item \textbf{Excluded middle}
    asserts that all propositions are either true (inhabited) or false (uninhabited);
    in Lean as \code{em} and in Rocq as \code{Coq.Logic.Classical\_Prop.classic}.
    This is equivalent to several other principles,
    including \textbf{double negation elimination}
    ($\forall (A : \kw{Prop})\mathpunct{.} \neg \neg A \to A$)
    and \textbf{Peirce's law}
    ($\forall (A \; B : \kw{Prop})\mathpunct{.} ((A \to B) \to A) \to A$).
  % \item \textbf{Choice}
\end{itemize}

Function extensionality, propositional extensionality, and univalence
are principles that extend propositional equality such that more things are equal.
Univalence together with proof irrelevance implies propositional extensionality,
since bi\"implicated propositions are isomorphic by irrelevance,
and univalence gives an equality from the isomorphism.
Univalence alone also implies function extensionality
by a more complex argument~\cite[Chapter 4.9]{hott}.

One application of these new equalities
may be encountered when encoding a function as a relation
whose functionality is proven \apost.
This is a frequent pattern in proof assistants,
as induction relations often have better ergonomic support
than dependently-typed reasoning over functions.
Consider an encoding of a two-place predicate
over a representation of types $\mathit{Ty}$ and terms $\mathit{Tm}$ ---
a function from $\mathit{Ty}$ to predicates $\mathit{Tm} \to \kw{Prop}$ ---
as a relation:
$$R : \mathit{Ty} \to (\mathit{Tm} \to \kw{Prop}) \to \kw{Prop}.$$
Such a relation could be a \emph{logical relation}~\citep{logrel}
used to model typed lambda calculi,
where a $\mathit{Ty}$ is interpreted as a set of $\mathit{Tm}$s.
Functionality of $R$ demonstrates that $\mathit{Ty}$s have unique interpretations.
To show that $R$ is functional, \ie
$$\forall (A : \mathit{Ty}) (P \; Q : \mathit{Tm} \to \kw{Prop}) \mathpunct{.}
  R \; A \; P \to R \; A \; Q \to P \equiv Q,$$
it suffices to show that
$\forall (a : \mathit{Tm})\mathpunct{.} P \; a \leftrightarrow Q \; a$,
since $\forall (a : \mathit{Tm})\mathpunct{.} P \; a \equiv Q \; a$
follows from propositional extensionality,
and finally $P \equiv Q$ from function extensionality.

The disadvantage of axiomatic equalities is that substitutions will not further reduce on them,
which can make reasoning about the substituted terms difficult.
There are type theories beyond MLTT and CIC that are designed
so that these principles are instead provable theorems,
such as cubical type theories~\citep{bch,cchm,afh,cartesian}
and Cubical Agda~\citep{cubical-agda} for univalence,
and observational type theory~\citep{ott,ott-now,ttobs,ccobs,cicobs}
for function and propositional extensionality.

Excluded middle, double negation elimination,
and other equivalent propositions are \emph{classical} reasoning principles
that do not hold intuitionistically.
Nevertheless, because a large majority of mathematics is done classically,
many communities mechanizing mathematics will freely use classical principles.
In particular, Lean's mathematical library mathlib~\citep{mathlib}
contains proofs that rely on classical axioms,
and tactics such as \code{tauto} will automatically apply classical reasoning.
There is work towards constructively integrating classical principles into type theory consistently,
typically in the form of calculi with control operators~\citep{mu,mumu,dl,sr}.

One has to be careful, however, that a chosen set of features and axioms
do not render the type theory logically inconsistent and thus useless for proving.
A number of them are known to be incompatible with one another;
below are a few such combinations.

\begin{itemize}[noitemsep,topsep=0pt]
  \item Strong elimination is inconsistent for large impredicative inductives.
    \citet{strong-pair} show that impredicative dependent pairs with pair projections,
    which can correspond to strong elimination,
    are inconsistent.
    \citet{trees} also demonstrates the inconsistency with an inductive type $U$
    with a single constructor of type
    $\forall (X : \kw{Prop})\mathpunct{.} (X \to U) \to U$.
  \item Strong elimination is also inconsistent for inductive propositions
    when $\kw{Prop}$ is proof irrelevant.
    As seen above, strong elimination suffices to show that $\neg (\kw{true} \equiv \kw{false})$.
    If $\kw{Bool}$ is defined in a proof-irrelevant $\kw{Prop}$,
    $\kw{true} \equiv \kw{false}$ would hold by definition,
    which is a contradiction.
  \item Strong elimination is once again inconsistent for inductive propositions
    in the presence of impredicativity and classical principles such as excluded middle.
    A modern implementation of the construction by \cite{em-irr},
    such as \code{Coq.Logic.Berardi} in Rocq's standard library,
    uses excluded middle to derive propositional proof irrelevance,
    which can be used as above to derive a contradiction.
  \item UIP is inconsistent with univalence.
    Intuitively, univalence produces an equality between two types
    given an equivalence between them,
    and there are types that are equivalent in multiple, provably different ways,
    so there are equalities between them that are provably different,
    thus violating UIP.
    Concretely, $\kw{Bool}$ is equivalent to itself in two different ways,
    either by mapping booleans to themselves or to their negation,
    so there are distinct proofs of $\kw{Bool} \equiv \kw{Bool}$
    \citep[Example 3.1.9]{hott}.
\end{itemize}

\begin{figure}[h]
\centering
\begin{tikzcd}
  & \mathcolor{red}{\varnothing} & & \\
  \cdot \equiv \cdot : \kw{SProp} \arrow[ru,dashed] & & \mathrm{UA} + \kw{SProp} \arrow[lu,dashed] & \\
  \mathrm{UIP} \arrow[u] & \kw{SProp} \arrow[ru] \arrow[lu] & \mathrm{propext} \arrow[u] & \mathrm{UA} \arrow[lu] \\
  & \kw{Prop} \arrow[u] \arrow[ru] & \mathrm{funext} \arrow[ru] & \\
  & \mathrm{base} \arrow[u] \arrow[luu] \arrow[ru] & &                       
\end{tikzcd}
\caption{A compatibility graph of theories with impredicative $\kw{Prop}$, proof irrelevance ($\kw{SProp}$),
  UIP, univalence (UA), function extensionality (funext), propositional extensionality (propext),
  and (in)compatible combinations.}
\label{fig:lattice}
\end{figure}

\Cref{fig:lattice} illustrates some of the aforementioned relationships between extensions.
The arrows point from one theory to a greater encompassing theory;
for instance, a theory with propositional extensionality extends the equalities
of a theory with a universe of propositions,
and a theory with univalence can derive function extensionality.
At the top of the graph, the dotted arrows indicate
the incompatibility of a theory that implies UIP with one that contains univalence:
there is no possible encompassing theory.

To prevent inconsistencies, features may be hidden behind option flags,
or disallowed entirely.
Rocq, Lean, and Agda all disallow strong elimination for inductive propositions
in $\kw{Prop}$ and $\kw{SProp}$,
with the exception of \emph{syntactic subsingletons},
which are inductives that syntactically have at most one inhabitant,
such as $\top$, $\bot$, and conjunction of propositions.
While Rocq's impredicative $\kw{Prop}$ universe is not proof irrelevant,
strong elimination is still forbidden even for inductives that aren't large
to allow the use of classical principles.
There is a compiler flag \code{-impredicative-set}
to enable impredicativity for $\kw{Set}$
while still allowing strong elimination of small impredicative inductives,
but this flag is no longer supported.
In Agda, the \code{-{}-with-K} flag enables Axiom K
and the \code{-{}-without-K} flag disables it,
while univalence can be proven as part of the features enabled by Cubical Agda
using the \code{-{}-cubical} flag.
There is also a \code{-{}-safe} flag which disallows,
among other combinations,
having both \code{-{}-with-K} and \code{-{}-cubical}.

There is also some limited ability in tracking the usage of features and axioms.
In Rocq, the axioms and unsafe flags used by a definition
can be listed using the command \code{Print Assumptions},
while in Lean, the axioms can be listed using \code{\#print axioms}.
In Agda, along with checking for inconsistent option combinations,
\code{-{}-safe} will also ensure the absence of any \code{postulate}s.
Since option flags can be enabled at module-level granularity,
there is also a notion of \emph{(co)infective} flags:
an infective flag used in one module
must be used by all modules that depend on that module,
while a coinfective flag used in one module
may only depend on modules that also use that flag.

While users of these proof assistants have been getting along fine
with these compiler tools for tracking features and axioms for the past decade or more,
there is plenty of room of improvement;
I have identified three shortcomings these systems.
%
\begin{itemize}
  \item There is no way of safeguarding against a feature or an axiom;
    that is, it cannot be guaranteed generally that a particular definition
    does \emph{not} use something.
    If a definition can be guaranteed not to use some given feature,
    then it is safe to be used by other definitions
    that are incompatible with that feature.
    For instance, ensuring that a proof does not use Axiom K
    means that it may be used by another proof that does use univalence.
  \item The scope of feature flags is too coarse.
    They range from project-level compiler flags
    down to module-level option flags,
    but even this level of granularity is not the appropriate one:
    modules are intended for organizing definitions by semantic content,
    rather than by the collection of features they happen to all use.
    This prevents reuse of a definition in one module inside another module
    if they happen to have incompatible features,
    even if that particular definition does not depend on any features at all.
  \item Tracking features and axioms is only a compiler tool
    and is not formally described along with the underlying type theories themselves.
    Furthermore, there is no formal description of the interactions between
    the different theories that result from including various features and axioms.
\end{itemize}

An ideal system for enabling and disabling features and axioms, then,
should track which ones are and aren't used, at least at the definition level,
and it should be formally described with the type theory as a whole.
This suggests that a system for dependency tracking would be a good starting point.
Therefore, I propose that the Dependent Calculus of Indistinguishability (DCOI)~\citep{dcoi},
a type system that incorporates dependency tracking and dependent types,
could be a suitable framework for tracking multiple type theories.

\section{A Primer on DCOI} \label{sec:dcoi}

DCOI is a Pure Type System (PTS)~\citep{pts} augmented with dependency tracking~\cite{dcc}.
An instantiation of DCOI's PTS rules and axioms governing how types and terms interact
is DCOI$^\omega$~\citep{dcoi-omega},
which has dependent types and a predicative universe hierarchy,
making it suitable as a foundation for theorem proving.
In a typing judgement, dependency tracking appears as annotations
on both the variables in the context,
to indicate how they may be used by the term being typed,
and alongside the type of the term,
to indicate how the term itself may be used.
As an example, consider the following derivable typing judgement for a constant function.
$$A :^\HH \kw{Type} \vdash \lambda x^\LL \; y^\HH \mathpunct{.} x :^\LL
  A^\LL \to A^\HH \to A$$
The concrete levels used here are low ($\LL$) and high ($\HH$) where $\LL < \HH$,
though DCOI is general over any meet-semilattice.
In the context of security flow, these levels correspond to low- and high-security computations
where low-security computations may not inspect the values of high-security ones.
They can also be thought of in terms of computational irrelevance,
where something marked as computationally irrelevant ($\HH$)
must not play a part in the execution of relevant programs ($\LL$),
and may even be erased away after compilation.

This constant function at low,
which returns its first argument $x$ and ignores its second argument $y$,
must therefore mark $x$ and its type as low to return it,
and marking $y$ and its type as high guarantees that it could not return it.
While the body of the low function cannot return a high argument,
its type \emph{can} depend on a high term,
demonstrated by the high-annotated type $A$ in the context,
which is used in the type of the function.
The intuition is that $A$ does not play a part in the run-time execution of the constant function,
but is otherwise permitted to participate in compile-time type checking.

There are a number of key aspects and properties of DCOI, covered shortly,
that highlight how dependency tracking interacts with terms and typing.
These are not comprehensive,
but are relevant to the desired applications of DCOI in the next section.

\paragraph{Relative relevance.}
The intuition of computational relevance and irrelevance
is not fixed to the low and high levels,
but is a relative concept between any two ordered dependency levels.
Suppose there is a super-high level $\SS$ such that $\LL < \HH < \SS$.
Then just as a low term may not meaningfully use a high term,
a high term also may not meaningfully use a super-high term.
The following derivable typing judgement demonstrates
how these three levels can interact.
$$P :^\HH \kw{Nat}^\SS \to \kw{Prop}, n :^\SS \kw{Nat} \vdash \lambda p^\LL \mathpunct{.} p :^\LL
  (P \; s^\SS)^\LL \to P \; (s + 1)^\SS$$
In the context, $P$ is a high predicate which takes as argument a super-high natural,
along with a super-high natural $n$.
Once again, the term being typed is a low function,
while higher terms are involved in its type.
Although the function is an identity function,
its domain and codomain types are syntactically different applications of $P$,
but this judgement still holds because the arguments of $P$ are super-high
and therefore irrelevant \emph{with respect to} $P$ at high.

\paragraph{Indistinguishability.}
In general, if $\ell_1 < \ell_2$, then at observer level $\ell_1$,
$f \; x^{\ell_2}$ must be definitionally equal to $f \; y^{\ell_2}$
regardless of what $x$ and $y$ are.
We say that they are \emph{indistinguishable} at level $\ell_1$.
This key equality is what permits the above example to type check,
since $P \; s^\SS$ is thus indistinguishable from $P \; (s + 1)^\SS$ at high.
Similarly, calling the constant function above $k$,
$k \; x^\LL \; y^\HH$ is indistinguishable from $k \; x^\LL \; z^\HH$ at low,
which expresses the idea that $k$ is truly constant in its second argument.

DCOI internalizes indistinguishability by indexing
its propositional equality type with an observer level.
In particular, the propositional equality
$k \; x^\LL \; y^\HH \equiv^\LL k \; x^\LL \; z^\HH$
is provable by reflexivity since the two sides are already indistinguishable
at low, the observer level of the equality.

\paragraph{Elimination of higher falsehoods.}
The principle that lower-level terms may not meaningfully depend
on higher-level terms means that,
just as lower-level functions may not return higher-level arguments,
destructors that return lower-level terms may not destruct higher-level terms.
This holds even if the term being destructed contains no inner information
(such as $\top$ or an equality proof),
since reducing the destruction on a constructor requires knowing
whether the term being destructed is a constructor at all.

The sole exception is the eliminator for $\bot$,
since it has no constructors, so there is no information to reveal.
The computational interpretation of having a proof of $\bot$ to eliminate
is that we have reached an impossible dead branch,
so what we do with it will never matter since it will never execute.
The ability of eliminate higher-level proofs of falsehood into lower-level terms
is useful when the type of a function rules out a particular branch,
and the function needs to be assigned a lower level than its type.

\paragraph{Subsumption and downgrading.}
While lower-level terms cannot inspect higher-level terms,
higher-level terms can inspect lower-level terms.
Furthermore, a lower-level term can be raised to a higher level by \emph{subsumption}:
if a term is well typed at level $\ell_1$,
then it is also well typed with the same type at a higher level $\ell_2 > \ell_1$.

However, if two terms are indistinguishable by some observer level $\ell_2$,
then they can be indistinguishable by a \emph{lower} observer level $\ell_1$ by \emph{downgrading}.
From a security flow perspective, the higher the observer level,
the more secure values may be observed,
so the more things are distinguishable,
since securer values will need to be compared as well instead of being ignored.
Going down an observer level means more things are being hidden away,
so more values will appear to be indistinguishable from one another.

\section{Lattices of Type Theories} \label{sec:lattice}

While the PTS rules and axioms of DCOI can be instantiated to produce different type systems,
each instantiation, such as DCOI$^\omega$,
is a fixed type system with a single set of terms and typing rules.
Towards the goal of integrating granular feature tracking into type theory itself,
this project poses the question:
\emph{What if different dependency levels corresponded to different type theories?}

More precisely, to track extensions on top of a base type theory,
we would begin with a bottom dependency level corresponding to this base.
Each new dependency level above bottom would contain one additional construct
corresponding to a new feature or axiom.
For instance, there could be an Axiom K eliminator that type checks
only at a level for UIP and above,
and there could be a builtin excluded middle axiom that type checks
only at a level for classical reasoning and above.

Because level annotations are part of contexts and typing judgements,
when a particular definition is safe to use is specified with precision,
which guarantees that a particular definition never exploits an extension without permission.
A definition that can be typed at the bottom level
would be safe to use at all levels by subsumption,
and guaranteed to never employ, say, classical reasoning.
Indistinguishability reflects this guarantee, as it asserts the property that
uses of values from higher forbidden theories can only trivial,
such as ignoring the value or passing it around uninspected.

As dependency levels form a meet-semilattice,
any two theories must have a meet,
which corresponds to only the constructs that they both have in common,
and which are therefore safe to use in either theory.
If the join of two theories exist,
then the constructs introduced in either one can be used at the joined level.
Crucially, not all joins exist; a UIP level cannot be joined with a univalence level,
since their co\"existence is contradictory.
The shape of the lattice depends on the compatibilities between theories,
as well as implication order of extensions,
since one theory that encompasses the consequences of another
can be placed above that other theory.
The compatibility graph in \cref{fig:lattice}
is an example of a concrete lattice of theories,
where the arrows point towards the greater theory
and indicate the direction in which definitions can be raised.

One condition on the individual theories is that they must each be logically consistent.
If an inconsistency exists at any theory,
by the elimination of higher falsehoods,
the inconsistency will infect all lower theories, including the bottom theory.
Then by subsumption, the inconsistency at the bottom theory
can be raised to infect all higher theories,
and the entire lattice will be inconsistent.
This means that any theory that features nontermination will not be permitted.

Another catch is that a theory whose extension is a new definitional equality
(\ie a new rule for indistinguishability) will also be ineffective.
Even if that equality is defined for a given observer level,
it will hold for all lower observer levels by downgrading,
and the extension will be available to all lower theories.
This effect cannot be mitigated using restrictive premises,
as violating downgrading will violate many other desirable properties,
including transitivity of definitional equality~\citep{dcoi-omega}.

An unusual property of using DCOI for tracking theories
is that the type of a term may itself be well typed
within a different theory from that of the term.
It's unclear what it means when, for instance,
a term in the base theory can be assigned a type that uses classical principles.

\subsection{Objectives}

This project should answer the following questions:

\begin{enumerate}
  \item \label[question]{item:extensions}
    What kinds of extensions would fit within this framework?
    Some broad classifications of extensions might be
    ones that add new type universes (\eg $\kw{SProp}$),
    ones that expand the rules for existing constructs
    (\eg impredicativity, strong elimination),
    ones that add new computational constructs with reduction rules
    (\eg Axiom K),
    and ones that add new axiomatic constructs without reduction rules
    (\eg function and propositional extensionality, excluded middle).
  \item \label[question]{item:model}
    How would a particular lattice of theories be modelled
    to show desirable properties such as logical consistency?
    Ideally, the technique used to model a particular lattice
    should be broadly applicable and sufficiently extensible
    to be applied to a different lattice without redoing all the work,
    so that adding more extensions remains sustainable.
  \item \label[question]{item:properties}
    Are there properties resulting from using DCOI
    that aren't expected of a feature tracking system,
    such as a term and its type using different sets of features?
    What are the consequences of these properties,
    and are they beneficial or detrimental?
\end{enumerate}

To answer these questions, the project would be divided into two portions.
The first is an implementation of a type checker for a specific lattice of type theories.
The lattice should contain a sufficiently diverse set of labels and their orders
to answer \cref{item:extensions}.
\Cref{fig:lattice} is a good place to start,
as it contains theories in different classifications with different interactions.

To evaluate the viability of such a type checker,
a standard library would be implemented to exercise all levels of the lattice.
The standard libraries of Rocq\footnote{\url{https://coq.inria.fr/distrib/current/stdlib/}},
Agda\footnote{\url{https://agda.github.io/agda-stdlib/master/}},
and Lean\footnote{\url{https://leanprover-community.github.io/mathlib4_docs/}}
are good sources for inspiration,
as many of their files use the features and axioms mentioned in \cref{sec:extensions}.
An implementation would also serve to verify which extensions are indeed invalid
by demonstrating the inconsistencies or ineffectivities they yield.

With an implementation, useability concerns can be explored,
such as level inference.
Annotating definitions and arguments with every single extension it uses
is an unreasonable burden on a practical proof assistant user,
and it may be possible to infer the annotations
either based on the syntactic constructs used
or on what set of features are required for successful type checking.

The second portion is a formalized and ideally mechanized proof of consistency.
Because consistency is a semantic property and depends on the strength
of the metatheory used to model the type theory,
the formalization should model a lattice with (at least at first)
only one level above the base theory, the simplest nontrivial lattice.
The focus would be on how to combine two different models of type theory,
not on accommodating as many as possible from the outset.

A sensible starting point would be the mechanization of DCOI$^\omega$~\cite{dcoi-omega},
which proves consistency and normalization of what would be the base theory in the lattice,
and picking a reasonable feature to extend it with.
However, this mechanization uses a syntactic logical relation
indexed by well-founded universe levels as its semantic model,
which may limit its extensibility;
it cannot be straightforwardly extended to accommodate impredicativity,
nor to accommodate typed definitional equality.
A viable solution to \cref{item:model} must overcome this limitation.

A possible alternative is to use \emph{syntactic} modelling~\citep{syntactic},
which would involve a type-preserving translation into another type theory
whose consistency is well established,
guaranteeing consistency of the original system.
While there exist syntactic models of other type theories \citep{sprop,ghosts}
with notions of irrelevance, which is one application of indistinguishability,
a syntactic model of dependency tracking with dependent types is unexplored.

The process of accomplishing these two portions of the project
should answer \cref{item:properties},
either by the implementation revealing unexpected examples that can or cannot be type checked,
or by metatheoretical properties that hold based on the modelling technique chosen.
Only once these properties are revealed will we know what further work can be done,
from augmenting the implementation closer to a practical proof assistant,
to proving more complex theorems like normalization and decidability of type checking,
or proving consistency for a larger lattice.

\section{Prior work} \label{sec:prior}

This project builds on prior work on DCOI~\citep{dcoi} and DCOI$^\omega$~\citep{dcoi-omega},
on both of which I am second author.
For the former paper,
I implemented a prototype type checker for DCOI augmented with inductive types
by extending the minimal dependent type checker \texttt{pi-forall}~\citep{pi-forall},
and wrote examples using the type checker and motivating examples for DCOI.
I also proved a few of the lemmas in the mechanization.
For the latter paper,
I wrote about half of the prose, mostly for the earlier sections,
and proved a few of the lemmas as well.
As part of an investigation toward incorporating a relational model for DCOI,
I mechanized a PER model for MLTT based on the logical relation
used to prove consistency of DCOI$^\omega$,
but ultimately the gap between MLTT and DCOI could not be bridged,
so this work does not appear in the final paper.

Outside of DCOI, I have worked on Stratified Type Theory (StraTT)~\citep{stratt},
which annotates typing judgements similarly to dependency tracking,
but the annotations are universe levels,
and restrictions on what levels may be used where enforces consistency
where traditionally it is enforced by disallowing type-in-type.
In other words, instead of stratifying universes into a hierarchy,
typing judgements themselves are stratified.
Although StraTT is not a dependency tracking system in the same way DCOI is,
it demonstrates that there may be multiple ways to retain usage information
that enforces desired properties such as consistency or irrelevance.
Even if the particular setup for DCOI turns out not to be suitable for this project,
it may be reasonable to instead explore a more StraTT-like structure.

\section{Related work} \label{sec:related}

\subsection{Two-level type theory}

\subsection{Modal type theory}

\subsection{Proof assistants}

\subsection{Applications of extensions}

% Related work:
% * 2LTT & modalities
% * examples of large mechanizations
% * Trellys uses modalities to track general recursion

\section{Conclusion}

\noindent
\bibliography{main}
\end{document}